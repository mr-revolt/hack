% -*- conding UTF-8 -*-
% 组织文本

\documentclass[UTF8]{ctexart}
\usepackage[OT2,T1]{fontenc}
\usepackage[utf8]{inputenc}


\title{组织文本}
\author{revolt}
\date{\today}

\bibliographystyle{plain} % 参考文献

\begin{document}

\maketitle
\tableofcontents

\section{52个字母}

从英文字母,扩展到拉丁文字母.除了英文之外,还有西文字体,这种字体看起来很是优美.


peu\quad importe ce qui m'arrivera\qquad demain je ne cesserai jamais de t'aimer. 

shelfful shelf{}ful shelf\/ful

\section{标点的使用}

''\,'A' or 'B?'\,'' he asked! % 单引号和双引号的分开来写

\section{各种符号的写法}

a punctuation dash --- like this.

Good: One,two,three\ldots

Bad: one,two,three...

she $\ldots$ she got it

I've no idea\ldots

\# \quad \$ \quad \% \quad \& \quad

\{ \quad \} \quad \_ \quad

\textbackslash % 下划线

\punctstyle{quanjiao}
\punctstyle{banjiao}
\punctstyle{kaiming} 
%\punctstyle{hangmoban} % 行末半角
\punctstyle{plain} % 无格式

\section{空格和换行}

Happy \TeX ing Happy \TeX\ ing

Happy \TeX{} ing Happy {\TeX} ing

% ~ 在latex中被称为 带子 ties.

Questions~1 % 名称和编号间

Donald~E, Knuth % 名字之间,姓可以断开

function~$(x)$ % 短公式的写法

1,~2,and ~3 %序列的部分符号间

Roman number XTT\@ Yes

Tubjer et al. \ made the double play

中文和English混排效果,这个应该 space 好看

\mbox{条目}-a 不同于条目 -b

小子\phantom{4}洗好脖子等死

小子洗好脖子等死

小子\hphantom{4}洗好脖子等死 %水平方向缩进

小子\vphantom{4}洗好脖子等死 %垂直方向缩进

这是一行文字 \\另外一行

% 这是一行文字\linebreak[4] 另外一行

\section{字体}


\textit{Italic font test}

{\bfseries Bold font test}

\textmd{Medium series}

\textbf{Bold extended series}

{\itshape M}M

\textit{M}m

{\itshape M\/}M

Bold '{\bfseries leaf}'

Bold '{\bfseries leaf\/}'

Bold '\textbf{leaf}'

\textit{M}M

\textit{M\nocorr}M

% \newcommand\nocorrlist{,.}

{\CJKfamily{hei}黑体}

{\CJKfamily{sourcecodepro}Source Code Pro}

{\CJKfamily{kai}楷体}

{\CJKfamily{song}宋体}

{\songti 宋体} \quad {\heiti 黑体} \quad {\kaishu 楷书} \quad {\fangsong 仿宋}

\begin{quote}

fontencoding{编码}

fontfamily{族}

fontseries{系列}

fontshape{形状}

fontsize{大小}{基本行距}

\end{quote}

\fontencoding{OT1}\fontfamily{pzc}
\fontseries{mb}\fontshape{it}
\fontsize{14}{17}\selectfont
PostScript New Century SchoolBook 

% \usefont{编码}{族}{系列}{形状}
\usefont{OT1}{pzc}{mb}{it} PostScript New Centry SchoolBook

\usefont{T1}{pbk}{db}{n} PostScript Bookman Demibold Normal
\end{document}